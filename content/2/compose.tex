Uno de los principales puntos que se han visto durante el desarrollo del documento
es la necesidad de comunicar contenedores y de distribuir aplicaciones en var­ias
imágenes Docker.

Un \textit{stack} típico de aplicaciones sería un servidor XAMPP: un contenedor
ejecutando Apache que se comunica con una base de datos MySQL y que utiliza
un \textit{backend} basado en PHP. Se puede construir una única imagen la cual
se componga de esas tres aplicaciones y sus dependencias. Sin embargo, según
se recomienda desde Docker, lo ideal para no añadir demasiada complejidad sería
crear un contenedor mínimo para cada aplicación que funcione de forma aislada del
resto pero que se comunique con ello.

Si bien esta tarea se podría realizar manualmente con los comandos que se
han visto anteriormente, existe una herramienta llamada ``\texttt{docker-compose}''
la cual gestiona clústers de contenedores que se ejecutan a la vez y que dependen
unos de otros.

\subsubsection*{¿Qué es Docker Compose?}
Docker Compose es una herramienta basada en Python que se usa para definir
aplicaciones multicontenedor dentro de Docker. Se basa en el formato de ficheros
YAML para definir los servicios de la aplicación y, con un único comando, 
preparar su ejecución y trabajar con dichos servicios \cite{OverviewDockerCompose2021}.

La definición de un servicio de Compose se define en tres pasos:

\begin{enumerate}
    \item Definir los ficheros \texttt{Dockerfile} de cada uno de los entornos de
          la aplicación.
    \item Definir qué servicios componen la aplicación en un fichero \texttt{docker-compose.yml}.
    \item Ejecutar el comando ``\lstinline[style=bash]!docker compose up!'' para lanzar
          la aplicación al completo.
\end{enumerate}

Un fichero \texttt{docker-compose.yml} tiene una forma como esta (código \ref{lst:compose}):

\begin{lstlisting}[style=docker-compose, caption={Estructura típica de un fichero de Docker Compose \cite{OverviewDockerCompose2021}.}, label={lst:compose}]
version: "3.9"  # optional since v1.27.0
services:
  web:
    build: .
    ports:
      - "5000:5000"
    volumes:
      - .:/code
      - logvolume01:/var/log
    links:
      - redis
  redis:
    image: redis
volumes:
  logvolume01: {}
\end{lstlisting}

La estructura que se sigue habitualmente se define por las siguientes claves
en el fichero YAML \cite{DockerComposeTutorial}:

\begin{itemize}
  \item \texttt{version: "XY"} define qué versión de Docker Compose se está 
        usando. Esta línea, aunque opcional desde la versión \texttt{v1.27.0},
        es fundamental ya que las características de Docker Compose varían según
        la versión que se esté usando. Es importante que en este caso una versión
        posterior no implica que ``sea mejor'' sino que algunas características
        cambian de nombre, otras aparecen y otras se eliminan.
  \item \texttt{services} -- esta sección define los distintos contenedores que
        se van a crear. En el ejemplo del código \ref{lst:compose}, se crean dos
        servicios: uno primero que es \texttt{web} que se construye sobre el
        directorio actual; y un segundo con nombre \texttt{redis} que usa la
        imagen de Redis desde Docker Hub.
  \item \texttt{build} especifica la ubicación del fichero \texttt{Dockerfile}. En
        en ejemplo del código \ref{lst:compose} se usa el directorio actual `\texttt{.}'
        como ubicación para construir el contenedor.
  \item \texttt{ports} especifica los puertos mapeados del contenedor en cuestión.
        Es el equivalente a la opción \lstinline[style=bash]!-p! del comando
        \texttt{docker}.
  \item \texttt{volumes} define los volúmenes que se van a usar en el contenedor.
        En particular es como usar la opción \lstinline[style=bash]!-v! a la hora
        de crear un contenedor. Por ende, se puede usar un directorio para hacer
        \textit{bind mounts}, usar el nombre de un volúmen, etc.
  \item \texttt{links} permite unir un contenedor a otro. En este caso, se especifica
        a qué contenedor se puede acceder.
  \item \texttt{image} permite definir un servicio en un \texttt{docker-compose}
        usando una imagen ya existente en Docker Hub, si no se dispone del
        \texttt{Dockerfile} apropiado.
  \item \texttt{environment} define las variables del entorno del contenedor
        y sus respectivos valores. Equivalente a la opción \lstinline[style=bash]!-e!
        del comando \texttt{docker}.
\end{itemize}

Así, con la estructura del fichero ya definida, usando el comando \texttt{docker-compose}
se puede \cite{OverviewDockerCompose2021}:
\begin{itemize}
    \item Iniciar, parar y reconstruir servicios.
    \item Ver el estado de los servicios en ejecución.
    \item Volcar los registros de ejecución en tiempo real (\textit{stream view}).
    \item Ejecutar un único comando en un servicio.
\end{itemize}

Tras esta idea se ha construido una forma muy potente y simple de crear aplicaciones
multicontenedor. Aprovechando las características de Docker, Compose permite 
ejecutar distintos entornos aislados en el mismo anfitrión. Por ejemplo, de esta
forma, una aplicación Docker Compose puede contar con varios entornos según
en donde se quiera ejecutar. Por ejemplo, se puede tener un entorno de desarrollo,
otro de CI y uno de producción en el mismo \textit{host}. Esto se consigue gracias
a los nombres de proyecto, característica fundamental de Docker Compose para
distinguir y aislar entornos.

Otro punto interesante es que se puede actualizar el fichero ``\texttt{docker-compose.yml}''
cuando se necesite y la construcción de los contenedores se realizará únicamente
para aquellos que hayan cambiado. Esto da una gran agilidad a la hora de 
presentar actualizaciones en un entorno de producción y potencia la idea de
microservicios, en donde solo se actualiza lo que se necesita.

Esto afecta también a cómo se gestionan los datos de los contenedores. Si el 
volumen asociado a un contenedor ya existía se sigue usando el mismo, en ningún
momento se ``tira abajo'' y se empieza desde cero.

Finalmente, una de las características más interesantes de Docker Compose es que
soporta las variables del entorno. Esto se traduce en que un mismo fichero
YAML de Docker Compose puede producir distintas configuraciones según el
entorno en que se encuentre. Aprovechando esta característica, se pueden definir
y usar ficheros \texttt{.env} para establecer las variables del entorno que
se usarán a la hora de desplegar los servicios y trabajar con los contenedores.

\subsubsection*{Comandos Docker Compose}
De entre todas las opciones que ofrece Compose, algunas son muy interesantes por
las potencias que ofrecen y sus utilidades \cite{DockerComposeTutorial}:

\begin{itemize}
  \item \lstinline[style=bash]!docker-compose build! prepara las imágenes 
        construyendo los servicios, listos para ser lanzados. Si una imagen
        es del repositorio \textit{online} no se hace nada.
  \item \lstinline[style=bash]!docker-compose images! lista las imágenes que
        se han construido desde el fichero actual.
  \item \lstinline[style=bash]!docker-compose stop! detiene los servicios en
        ejecución actuales.
  \item \lstinline[style=bash]!docker-compose run <service>! crea los contenedores
        para el servicio especificado.
  \item \lstinline[style=bash]!docker-compose up! inicia todo el proceso de despliegue
        de los servicios: primero, construye las imágenes necesarias y luego
        inicia cada uno de los contenedores especificados.
  \item \lstinline[style=bash]!docker-compose ps! lista todos los contenedores del
        \texttt{docker-compose.yml} actual.
  \item \lstinline[style=bash]!docker-compose down! detiene todos los contenedores
        y limpia sus datos, redes e imágenes.
\end{itemize}

\subsubsection*{Caso real}
A modo de demostración, se va a crear un \textit{stack} Wordpress, NGINX, PHP-FPM
y MySQL para desplegar un servidor web Wordpress de forma sencilla (basándose en
el ejemplo de Digital Ocean \cite{ComoInstalarWordPress}).

Asumiendo que los ficheros de configuración de NGINX ya están preparados así
como las dependencias necesarias, se pasa a definir los ficheros de Docker
Compose. Por una parte, para la base de datos, se van a definir las
credenciales en un fichero \texttt{.env}:

\begin{lstlisting}[style=bash, caption={}]
MYSQL_ROOT_PASSWORD="password-random"
MYSQL_USER="wordpress_user"
MYSQL_PASSWORD="wordpress-random-password"
\end{lstlisting}

A continuación, se pasa a definir cada uno de los servicios que compondrá la
aplicación. El primero de ellos será la base de datos MySQL (código \ref{lst:compose-mysql}):

\begin{lstlisting}[style=docker-compose, caption={Servicio de MySQL para el \textit{stack} Wordpress.}, label={lst:compose-mysql}]
version: '3'

services:
  db:
    image: mysql:8.0
    container_name: db
    restart: unless-stopped
    env_file: .env
    environment:
      - MYSQL_DATABASE=wordpress
    volumes:
      - dbdata:/var/lib/mysql
    command: '--default-authentication-plugin=mysql_native_password'
    networks:
      - app-network
\end{lstlisting}

Lo primero que se especifica es la versión de Docker Compose que se necesita, en
este caso, la $3$. Para este caso en particular, se va a usar la imagen de
MySQL versión 8.0 (clave \texttt{image}) con nombre ``\texttt{db}'' y que se
debe reiniciar hasta que se detenga (clave \texttt{restart}). En lo referente
a las variables del entorno, se define el nombre de la base de datos asignando
la clave \texttt{MYSQL\_DATABASE} y se define un volumen ``\texttt{dbdata}'' el
cual almacenará los datos de MySQL de forma persistente. Finalmente, se especifican
algunas opciones iniciales al comando inicial de MySQL y se especifica la red
en la que estará conectada.

A continuación, se define el servicio de Wordpress (código \ref{lst:compose-wordpress}):

\begin{lstlisting}[style=docker-compose, caption={Servicio de Wordpress.}, label={lst:compose-wordpress}]
  wordpress:
    depends_on:
      - db
    image: wordpress:5.1.1-fpm-alpine
    container_name: wordpress
    restart: unless-stopped
    env_file: .env
    environment:
      - WORDPRESS_DB_HOST=db:3306
      - WORDPRESS_DB_USER=$MYSQL_USER
      - WORDPRESS_DB_PASSWORD=$MYSQL_PASSWORD
      - WORDPRESS_DB_NAME=wordpress
    volumes:
      - wordpress:/var/www/html
    networks:
      - app-network
\end{lstlisting}

En este caso, la configuración es similar al de MySQL pero se añade una directiva
que define el orden de inicio de los contenedores (\texttt{depends\_on}). Se va a
usar la imagen de \texttt{wordpress:5.1.1-fpm-alpine} que es la versión de Wordpress
\texttt{5.1.1} con PHP-FPM por detrás en la versión \textit{alpine}, derivada del
proyecto Alpine Linux que busca un tamaño reducido de las imágenes. A continuación,
se especifica el fichero que contiene las variables del entorno (\texttt{env\_file})
y se definen las opciones de Wordpress. Finalmente, se crea un volúmen que contendrá
los datos de Wordpress y se asocia al directorio \texttt{/var/www/html} dentro
del contenedor y se define la red a la que se conecta.

Por último, pero no menos importante, se define el servidor web en sí. En este caso,
se usa NGINX (código \ref{lst:compose-nginx}):

\begin{lstlisting}[style=docker-compose, caption={Servicio de NGINX.}, label={lst:compose-nginx}]
  webserver:
    depends_on:
      - wordpress
    image: nginx:1.15.12-alpine
    container_name: webserver
    restart: unless-stopped
    ports:
      - "80:80"
    volumes:
      - wordpress:/var/www/html
      - ./nginx-conf:/etc/nginx/conf.d
    networks:
      - app-network
\end{lstlisting}

De la configuración anterior es importante fijarse en que, por una parte, se
expone el puerto \texttt{80} (HTTP) al exterior (opción \texttt{ports}) y se
reutiliza el volúmen de Wordpress. Además, se hace un \textit{bind mounts} del
fichero ``\texttt{nginx-conf}'' dentro del directorio \texttt{/etc/nginx/conf.d}.
Esto permite realizar cambios en la configuración de NGINX y que se puedan aplicar
directamente.

Finalmente, al final del fichero, se añade la especificación de la red y de
los volúmenes (código \ref{lst:compose-end}):

\begin{lstlisting}[style=docker-compose, caption={Especificación de los volúmenes y de las redes.}, label={lst:compose-end}]
volumes:
  wordpress:
  dbdata:

networks:
  app-network:
    driver: bridge
\end{lstlisting}

Quedando un fichero \texttt{docker-compose.yml} así:

\begin{lstlisting}[style=docker-compose, caption={}]
version: '3'

services:
  db:
    image: mysql:8.0
    container_name: db
    restart: unless-stopped
    env_file: .env
    environment:
      - MYSQL_DATABASE=wordpress
    volumes:
      - dbdata:/var/lib/mysql
    command: '--default-authentication-plugin=mysql_native_password'
    networks:
      - app-network

  wordpress:
    depends_on:
      - db
    image: wordpress:5.1.1-fpm-alpine
    container_name: wordpress
    restart: unless-stopped
    env_file: .env
    environment:
      - WORDPRESS_DB_HOST=db:3306
      - WORDPRESS_DB_USER=$MYSQL_USER
      - WORDPRESS_DB_PASSWORD=$MYSQL_PASSWORD
      - WORDPRESS_DB_NAME=wordpress
    volumes:
      - wordpress:/var/www/html
    networks:
      - app-network

  webserver:
    depends_on:
      - wordpress
    image: nginx:1.15.12-alpine
    container_name: webserver
    restart: unless-stopped
    ports:
      - "80:80"
    volumes:
      - wordpress:/var/www/html
      - ./nginx-conf:/etc/nginx/conf.d
    networks:
      - app-network

volumes:
  wordpress:
  dbdata:

networks:
  app-network:
    driver: bridge
\end{lstlisting}

Después de realizar varias configuraciones iniciales, el servidor Wordpress ya
estará completamente funcional y disponible (ver el ejemplo completo en la
web de Digital Ocean \cite{ComoInstalarWordPress}).
