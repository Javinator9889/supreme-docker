Ahora que ya se han introducido los contenedores, las tecnologías de virtualización
y tendencias de uso, se va a explicar cómo funciona Docker en profundidad. Por una
parte, se va a ver cómo es la estructura de un contenedor Docker, cómo se
comunica con el kernel de Linux, cómo se aísla del resto del sistema y cómo
funciona a nivel de discos virtuales, interfaces de red y gestión de recursos.

Por otra parte, se comentarán diversos ejemplos y estructuras básicas que permiten
la creación de un contenedor aislado, la comunicación de varios contenedores y
el despliegue de una aplicación basada en múltiples contenedores funcionando
simultáneamente.

Finalmente, se comentarán tecnologías de orquestación de contenedores, como son los
clústers de Kubernetes y Docker Swarm y qué planes hay previstos de cara al desarrollo e
innovación de Docker como cliente y gestor de contenedores, para dar pié a un
análisis de la seguridad real de los contenedores, en el punto \ref{sec:security}.