La creación de un contenedor siempre se realiza de la misma manera: mediante un
fichero \texttt{Dockerfile}. Los \texttt{Dockerfile} son ficheros del estilo de
los \texttt{Makefile} que contienen unas reglas básicas que definen lo que será
una imagen de un contenedor.

El comando que crea una imagen es \lstinline[style=bash]!docker build!, el cual
toma las instrucciones que aparecen en el fichero y las va ejecutando una a una
hasta que se produce un error o finaliza correctamente.

Un fichero \texttt{Dockerfile} cuenta con multitud de directivas que permiten
personalizar y configurar cómo va a funcionar \cite{DockerfileReference2021}.
Aquí se van a introducir algunas de las más importantes (o las más usadas) \cite{jethvaHowDockerfileWorks}:

\begin{itemize}
    \item \texttt{FROM}: define una imagen base de partida sobre la que construir.
    \item \texttt{ADD}: copia ficheros y directorios dentro de la imagen del 
          contenedor. Además, acepta URLs como parámetro y las descarga
          directamente dentro.
    \item \texttt{RUN}: añade capas a la imagen base, instalando aplicaciones,
          librerías y componentes.
    \item \texttt{CMD}: especifica qué comandos se deben ejecutar al iniciarse
          el contenedor. Es importante tener en cuenta que solo puede existir una
          instrucción \texttt{CMD} en el fichero \texttt{Dockerfile} (si no, se usa solo
          el último que aparezca). Su funcionalidad principal
          es la de establecer los valores por defecto de un contenedor en ejecución.
    \item \texttt{ENTRYPOINT}: es la instrucción ``principal'' de un \texttt{Dockerfile}.
          Especifica el punto de entrada de un contenedor que se quiere que funcione como
          un ejecutable. Por ejemplo, el siguiente comando ``\lstinline[style=bash]!docker run -it --rm -p 80:80 nginx!''
          ejecuta un servidor NGINX de forma interactiva, publica el puerto 80 y, cuando finaliza
          su ejecución, se elimina.
    \item \texttt{ENV}: define las variables del entorno del contenedor que se usarán
          en tiempo de ejecución.
    \item \texttt{COPY}: con una funcionalidad similar a \texttt{ADD} pero con
          más limitaciones.
    \item \texttt{EXPOSE}: define en qué puertos la aplicación estará escuchando.
    \item \texttt{USER}: especifica el UID (o el nombre del usuario) que se usará
          internamente en el contenedor para ejecutar las aplicaciones.
    \item \texttt{VOLUME}: define volúmenes de datos a crear o un punto de montaje del
          contenedor en tiempo de ejecución.
    \item \texttt{WORKDIR}: especifica la ubicación en donde se ejecutará el comando
          en tiempo de ejecución.
    \item \texttt{LABEL}: especifica etiquetas del contenedor en forma de metadatos.
          Se conforman de parejas clave--valor que especifican información relativa
          a la imagen dentro del contenedor como, por ejemplo, la versión, el nombre
          del paquete, etc.
\end{itemize}

Por lo general, un \texttt{Dockerfile} comienza con la especificación de una imagen
de partida (ya que no es habitual crear una imagen desde cero) y, a continuación, se
procede a instalar diversos paquetes y capas que puedan ser necesarias para nuestra
aplicación. A continuación se copian los paquetes y requisitos de la aplicación
dentro de la imagen y se ejecuta la compilación o preparación del paquete, si
fuera necesario. Finalmente, se especifica el punto de entrada del contenedor y
los argumentos que recibirá, si recibe. Además, si es necesario se añaden los puertos
expuestos y volúmenes necesarios para el funcionamiento.

En el código \ref{lst:dockerfile-example} se tiene un ejemplo de un \texttt{Dockerfile}
completo \cite{BestPracticesWriting2021}:

\begin{lstlisting}[style=Dockerfile, caption={Ejemplo de \texttt{Dockerfile} para una aplicación Go \cite{BestPracticesWriting2021}.}, label={lst:dockerfile-example}]
# syntax=docker/dockerfile:1
FROM golang:1.16-alpine AS build

# Install tools required for project
# Run `docker build --no-cache .` to update dependencies
RUN apk add --no-cache git
RUN go get github.com/golang/dep/cmd/dep

# List project dependencies with Gopkg.toml and Gopkg.lock
# These layers are only re-built when Gopkg files are updated
COPY Gopkg.lock Gopkg.toml /go/src/project/
WORKDIR /go/src/project/
# Install library dependencies
RUN dep ensure -vendor-only

# Copy the entire project and build it
# This layer is rebuilt when a file changes in the project directory
COPY . /go/src/project/
RUN go build -o /bin/project

# This results in a single layer image
FROM scratch
COPY --from=build /bin/project /bin/project
ENTRYPOINT ["/bin/project"]
CMD ["--help"]
\end{lstlisting}

En el fichero anterior se establece una imagen de partida \texttt{golang:1.16-alpine}
que contiene los binarios de Go en una distribución de muy poco peso (proyecto Linux
Alpine). Esta imagen se obtiene desde \href{https://hub.docker.com/_/golang}{Docker Hub}.

A continuación, instala el paquete \texttt{git} y el gestor de dependencias de Go.
Después, copia el proyecto en sí e instala las dependencias en la ruta \texttt{/go/src/project}.
Finalmente, copia el directorio actual en dicha ruta y define una nueva imagen partiendo
de la de Golang en donde se ejecuta el proyecto compilado con las opciones
\texttt{-{}-help} por defecto.

Cuando se define un contenedor es recomendable instalar solo aquellas dependencias
que sean necesarias. Esto permite una mayor y mejor mantenibilidad, reduce la complejidad
y el tiempo de construcción de la imagen.

Por otra parte, se recomienda encarecidamente desacoplar la aplicación: por ejemplo,
un servidor web posiblemente requiera de varias aplicaciones en ejecución. Es recomendable
separarlas en contenedores y habilitar la comunicación entre ellos a encapsularlo todo
en un único contenedor. Esto facilita, entre otros, un escalado horizontal en donde
si se requieren de más imágenes se crean.

Además, el número de capas se debe mantener lo más pequeño posible: las instrucciones
\texttt{RUN}, \texttt{COPY} y \texttt{ADD} crean capas, mientras que otras instrucciones
solo crean imágenes intermedias.

Finalmente, es importante construir el \texttt{Dockerfile} teniendo en cuenta que Docker
usa una caché interna. Si se quiere deshabilitar se puede hacer con la opción
\lstinline[style=bash]!--no-cache=true! en el comando \lstinline[style=bash]!docker build!.
Sin embargo, es recomendable hacer uso de la caché interna de Docker ya que agiliza
el proceso de construcción.