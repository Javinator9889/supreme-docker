Docker es una tecnología muy en boga hoy en día. Sin embargo, varios problemas
durante el desarrollo y la falta de actualizaciones con características
demandadas han llevado al ``rey'' a caerse.

Parece que en los últimos meses (a junio de 2021) se han puesto las pilas y han
retomado con fuerza el desarrollo de su motor de ejecución y herramienta de
orquestación. Uno de los problemas que se encuentran muchas empresas a la hora
de desplegar contenedores es su naturaleza efímera y no persistente que, si bien
es cierto es su gran ventaja, muchas veces puede jugar en su contra.

Sería necesario un modelo de contenedor sostenible en el tiempo, que se mantenga
por sí mismo y que no requiera de atención por parte del gestor. Parece ser que
cada vez la tendencia tiene más hacia allí y Docker está enfocando sus esfuerzos
en escuchar a la comunidad y mantenerse como líder.

Otro factor crítico fue la noticia de \textit{deprecation} de Docker dentro de
Kubernetes \autocite{DonPanicKubernetes2020}: en favor del estándar de contenedores
\texttt{containerd} las imágenes nativas de Kubernetes se sustituyen por aquellas
basadas en CRI (\textit{Container Runtime Interface}). Si bien esto no deja de lado
a Docker ya que se sigue pudiendo ejecutar en Kubernetes, es un golpe duro ya que
las empresas están centrando sus esfuerzos en otras tecnologías de contenedores.

Es por ello por lo que Docker está apostando en su tecnología de orquestación, 
Swarm, para intentar imponerse como nuevo líder en gestión de contenedores. Es
una tarea compleja (principalmente porque Kubernetes es el líder) pero tiene
muchos factores a su favor, como la fácil integración y la sencillez en las
operaciones.