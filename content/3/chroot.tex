Desde las primeras versiones de Linux, el concepto de ``enjaular'' ya existía.
Crear un \texttt{chroot} se traduce en que una aplicación se ejecuta sobre
un directorio definido como si dicho directorio fuese la raíz. Por ejemplo,
se crea un \texttt{chroot} en la ruta \texttt{/home/user/jail}. Cualquier
proceso que se ejecute dentro de dicha ruta pensará que se está ejecutando
sobre la raíz \texttt{/}, y no tendrá acceso al sistema de ficheros que haya
por encima.

Esta característica fue uno de los primeros pasos en la creación y configuración
de \textit{sandboxes}, y además añadía cierta portabilidad porque siempre se
podía comprimir dicho directorio, llevarlo a otro equipo y trabajar sobre
sus contenidos directamente. Además, como en ese directorio el árbol UNIX
no se respeta del todo, es posible aislar un proceso del resto
excluyendo el directorio \texttt{/proc} o \texttt{/run} de la jaula.

¿Qué diferencias existen con Docker? A lo largo de todo el documento, ya
se tiene una idea muy clara de qué permite Docker. Principalmente, Docker presenta
un mecanismo de aislamiento mucho más sofisticado. En una jaula es necesario
renunciar a ciertos permisos o características para poder trabajar con
seguridad. A veces es posible que no sea sencillo ejecutar un proceso o aplicación
en una jaula y la configuración que se diseñe podría dañar directamente al sistema,
ya que una jaula siempre se ejecuta como administrador. Por su parte, el aislamiento
con Docker se consigue mediante los \textit{namespaces} de Linux.

Por otra parte, en Docker los procesos tiene su propia tarjeta de red, su propio
identificador (\textit{hostname}), memoria dedicada, limitación de recursos, \dots
Es decir, es una solución mucho más sofisticada.

Docker se compara más a una máquina virtual ya que permite usar un sistema
distinto al que está instalado, mientras que en una jaula se comparte
todo del sistema anfitrión. Por ende, la característica principal que diferencia una
solución de otra es la capacidad de aislar procesos, recursos, memoria y red.