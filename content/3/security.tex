A lo largo de todo el documento se ha visto cómo la tecnología de contenedores
ha enfocado gran parte de su esfuerzo en la seguridad: desde el diseño inicial
hasta la implementación final las características de seguridad de los 
contenedores han supuesto un punto de inflexión en estas tecnologías.

La cuestión es: ¿cuánto de seguros son? En común todas las tecnologías tienen
un ``fallo'' de base: requieren de permisos de administrador para ejecutarse.
Eso o bien el usuario que gestiona los contenedores pertenece a un grupo
privilegiado, que sería equivalente a ejecutar ``todos'' los comandos como
administrador.

En esta sección se va a tratar por una parte la seguridad en Docker desde el 
punto de vista de más bajo nivel, como es la comunicación con el kernel; hasta
aspectos más ``elevados'' como son diferencias con otras soluciones como
\texttt{chroot}, cómo afecta el \textit{firewall} a la seguridad de los
contenedores y cómo de seguras son las comunicaciones entre contenedores.